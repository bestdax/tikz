%!tex program = lualatex
\documentclass{ctexbook}
\usepackage[outputdir=build]{minted}
\usepackage[a4paper, margin=2cm]{geometry}
\usepackage{amsmath}
\markright{dax}
\usepackage{pyluatex}
\usepackage{tikz}
\usepackage{etoolbox}
\usetikzlibrary{
	arrows.meta,
	backgrounds,
	bending,
	calc,
	calendar,
	decorations.markings,
	decorations.pathmorphing,
	decorations.pathreplacing,
	decorations.text,
	external,
	fadings,
	fit,
	intersections,
	patterns,
	positioning,
	quotes,
	shadings,
	shadows,
	shapes.misc,
}
\usepackage{tcolorbox}
\tcbuselibrary{
	listings,
	skins,
	raster,
	breakable,
	minted,
	hooks,
	skins,
}

\usepackage{silence}
\WarningFilter{latexfont}{Font shape}
\WarningFilter{latexfont}{Some font shapes}
\WarningFilter{latexfont}{Size substitutions}
\newenvironment{ditemize}{\begin{tcbitemize}[raster columns=4, boxrule=0.2pt, colframe=green!50!black,
			colback=green!5]}{\end{tcbitemize}}

\newtcblisting{texlst}
{
	bicolor,
	boxrule=0.2pt,
	colback=green!5,
	colbacklower=yellow!15,
	colframe=gray!60,
	listing engine=minted,
	minted language=latex,
	minted options app={fontsize=\footnotesize, linenos, numbersep=5pt},
	righthand width=5cm,
	sidebyside,
}
\NewTotalTCBox{\commandbox}{ s v }
{verbatim,colupper=black!85,colback=gray!5,colframe=white}
{\IfBooleanT{#1}{\textcolor{black!85}{\ttfamily\bfseries \$>}}\lstinline[language=sh,keywordstyle=\color{blue!35!white}\bfseries]^#2^}

\newmintinline[texminted]{tex}{}
\newcommand{\texinline}[1]{\begin{tikzpicture}[baseline=(code.base)]
		\node[inner sep=2pt, text height=8pt, text depth=2pt, rounded corners, fill=green!5, draw=green!50!black](code){\texminted{#1}};
		\useasboundingbox ([xshift=-1pt]current bounding box.west) rectangle ([xshift=1pt]current bounding box.east);
	\end{tikzpicture}}

\newcommand{\tikzname}{Ti\textit{k}Z}

\newenvironment{arrowexamples}
{\begin{tabbing}
		\hbox to \dimexpr\linewidth-5.5cm\relax{\emph{Appearance of the below at line width} \hfil} \=
		\hbox to 1.9cm{\emph{0.4pt}\hfil} \= \hbox to 2cm{\emph{0.8pt}\hfil} \= \emph{1.6pt} \\
		}
		{\end{tabbing}\vskip-1em}

\makeatletter
\def\arrowexample#1[#2]{\def\temp{#1}\ifx\temp\pgfutil@empty\arrowexample@\currentarrowtype[{#2}]\else\arrowexample@#1[{#2}]\fi}
\def\arrowexample@#1[#2]{%
{\sfcode`\.1000\small\texttt{#1[#2]}} \>
\tikz [baseline,>={#1[#2]}] \draw [line
	width=0.4pt,->] (0,.5ex) -- (2em,.5ex); thin \>
\tikz [baseline,>={#1[#2]}] \draw [line
	width=0.8pt,->] (0,.5ex) -- (2em,.5ex); \textbf{thick} \>
\tikz [baseline,>={#1[#2]}] \draw [line
	width=1.6pt,->] (0,.5ex) -- (3em,.5ex); \\
}

\def\arrowexampledouble#1[#2]{\def\temp{#1}\ifx\temp\pgfutil@empty\arrowexampledouble@\currentarrowtype[{#2}]\else\arrowexampledouble@#1[{#2}]\fi}
\def\arrowexampledouble@#1[#2]{%
{\sfcode`\.1000\small\texttt{#1[#2]} on double line} \>
\tikz [baseline,>={#1[#2]}]
\draw [double equal sign distance,line width=0.4pt,->] (0,.5ex) -- (2em,.5ex); thin \>
\tikz [baseline,>={#1[#2]}]
\draw [double equal sign distance,line width=0.8pt,->] (0,.5ex) -- (2em,.5ex); \textbf{thick} \>
\tikz [baseline,>={#1[#2]}]
\draw [double equal sign distance, line width=1.6pt,->] (0,.5ex) -- (3em,.5ex); \\
}

\newenvironment{arrowcapexamples}
{\begin{tabbing}
		\hbox to \dimexpr\linewidth-5.5cm\relax{\emph{Appearance of the below at line width} \hfil} \=
		\hbox to 1.9cm{\emph{1ex}\hfil} \= \hbox to 2cm{\emph{1em}\hfil} \\
		}
		{\end{tabbing}\vskip-1em}

\def\arrowcapexample#1[#2]{\def\temp{#1}\ifx\temp\pgfutil@empty\arrowcapexample@\currentarrowtype[{#2}]\else\arrowcapexample@#1[{#2}]\fi}
\def\arrowcapexample@#1[#2]{%
{\sfcode`\.1000\small\texttt{#1[#2]}} \>
\kern-.5ex\tikz [baseline,>={#1[#2]}] \draw [line
	width=1ex,->] (0,.5ex) -- (2em,.5ex);  \>
\kern-.5em\tikz [baseline,>={#1[#2]}] \draw [line
	width=1em,->] (0,.5ex) -- (2em,.5ex);  \\
}
\catcode`\|=12
\gdef\pgfmanualnormalbar{|}
\catcode`\|=13
\AtBeginDocument{\gdef|{\ifmmode\pgfmanualnormalbar\else\expandafter\verb\expandafter|\fi}}


\makeatother

\includeonly{arrows}

\begin{document}
% \tableofcontents
\chapter{坐标}
一般常用的坐标系统有迪卡尔坐标、极坐标,这两种\tikzname
都提供了。可以用明确的方式指定坐标系统,也可以隐式的让\tikzname 自己推导,因为两种坐标的格式是不一样的。
\begin{texlst}
	\begin{tikzpicture}
		\draw[help lines] (0,0) grid (3, 2);
		\draw(canvas cs:x=0cm, y=2mm) -- (canvas polar cs: radius=2cm, angle=30); % 显式
		\draw[red](1, 2mm) -- (30:3cm); % 隐式
	\end{tikzpicture}
\end{texlst}

坐标如果不带单位的话,默认为cm。坐标可以是$(x, y)$式的,也可以用名称,但是所有的都需要用小括号给包住。

如果需要将坐标偏移的话,需要用\texinline{([xshift=3pt] 1,1)}这种写法。

坐标支持不同单位的四则运算。
\begin{texlst}
	\begin{tikzpicture}
		\draw[help lines] (0,0) grid (3,2);

		\fill(canvas cs:x=1cm, y=1.5cm) circle (2pt);
		\fill(canvas cs: x=1.5 * 2cm, y=-5mm + 2pt) circle (2pt);
	\end{tikzpicture}
\end{texlst}

\section{xyz坐标系统}
\begin{texlst}
	\begin{tikzpicture}[->]
		\draw (0,0) -- (xyz cs:x=1) node[right]{$x$};
		\draw (0,0) -- (xyz cs:y=1)node[above]{$y$};
		\draw (0,0) -- (xyz cs:z=1)node[below left]{$z$};
	\end{tikzpicture}
\end{texlst}
xyz坐标系统也可以采用隐式标记的方法来书写。
\begin{texlst}
	\begin{tikzpicture}[->]
		\draw (0,0) -- (1,0);
		\draw (0,0) -- (0,1,0);
		\draw (0,0) -- (0,0,1);
	\end{tikzpicture}
\end{texlst}
此外,xyz坐标系统也支持极坐标,不过,这一块我就不考虑了。

\section{重力坐标系统barycentric}
这种坐标系统可以看到权重的偏向,一般平时也用不到。

\begin{texlst}
	\begin{tikzpicture}
		\node(c)[text width=3cm, text height=3cm, draw]{};
		\coordinate(one) at (c.north east);
		\coordinate(two) at (c.south east);
		\coordinate(three) at (c.north west);
		\coordinate(four) at (c.south west);

		\node[above=0.5cm, text width=1cm, font=\tiny, align=center] at (one){重要\\紧急};
		\node[below=0.5cm, text width=1cm, font=\tiny, align=center] at (two){紧急\\不重要};
		\node[above=0.5cm, text width=1cm, font=\tiny, align=center] at (three){重要\\不紧急};
		\node[below=0.5cm, text width=1cm, font=\tiny, align=center] at (four){不重要\\不紧急};

		\node[font=\tiny, text width=1cm] at (barycentric cs:one=1,two=0,three=0,four=0){生命受到威胁};
		\node[font=\tiny, text width=1cm] at (barycentric cs:one=0,two=1,three=0,four=0){拉肚子上厕所};
		\node[font=\tiny, text width=1cm] at (barycentric cs:one=0,two=0,three=1,four=0){保持健康};
		\node[font=\tiny, text width=1cm] at (barycentric cs:one=0,two=0,three=0.5,four=1){找朋友喝酒};
	\end{tikzpicture}
\end{texlst}
这个重力坐标图的例子没有举好,不过大概就是这么个意思。

\section{节点坐标系统}
这种坐标系统就是用节点的名字来代替实际的数字坐标。
\begin{texlst}
	\usetikzlibrary{arrows.meta}
	\begin{tikzpicture}[every node/.style={scale=0.5}, scale=0.4]
		\node (shape) at (0,2) [draw] {class Shape};
		\node (rect) at (-2,0) [draw] {class Rectangle};
		\node (circle) at (2,0) [draw] {class Circle};
		\node (ellipse) at (6,0) [draw] {class Ellipse};

		\draw (node cs:name=circle,anchor=north) |- (0,1);
		\draw (node cs:name=ellipse,anchor=north) |- (0,1);
		\draw [arrows=-{Triangle[open, angle=60:1mm]}] (node cs:name=rect,anchor=north) |- (0,1) -| (node
		cs:name=shape,anchor=south);
	\end{tikzpicture}
\end{texlst}
还可以使用角度来替代锚点。
\begin{texlst}
	\usetikzlibrary{shapes.geometric}
	\begin{tikzpicture}
		\node (start) [draw, shape=ellipse] {start};
		\foreach \ang in {-90, -80, ..., 90}
		\draw (node cs:name=start,angle=\ang) .. controls +(\ang:1cm) and +(-1, 0) .. (2.5,0);
	\end{tikzpicture}
\end{texlst}
如果锚点和角度都没有提供的提供的话,\tikzname\ 会自动计算出一个合适的连接方式。
在实际的使用中,可以隐式的表达节点坐标系统,比如\texinline{(nodename.south)}或者\texinline{(nodename.30)}。

\section{正切坐标系统}
此坐标系统需要加载\texinline{calc}库,而且没有隐式的调用方法。
\begin{texlst}
	\usetikzlibrary{calc}
	\begin{tikzpicture}
		\draw[help lines] (0,0) grid (3,2);
		\coordinate(a) at (3,2);

		\node[circle, draw] (c) at (1,1) [minimum size=40pt] {$c$};

		\draw[red, thick] (a) -- (tangent cs:node=c,point={(a)},solution=1) -- (c.center) -- (tangent
		cs:node=c,point={(a)},solution=2) -- cycle;
	\end{tikzpicture}
\end{texlst}

手册里面还提到了一种自定义坐标系统的方法,不过我没有看懂,这个也不是目前的重点,先略过。

\section{交点}
\texinline{(2,1 |\textbar|- 3,4)}这种写法可以得到从\texinline{(2,1)}先垂直后水平方向得到交点。如下面图标:
\begin{texlst}
	\begin{tikzpicture}[scale=0.7]
		\draw[help lines](0,0)grid(3,4);
		\fill(2,1) circle (1pt);
		\fill(3,4) circle (1pt);
		\fill[red](2,1 |- 3,4) circle (1pt);
	\end{tikzpicture}
\end{texlst}

还有一种方法是需要载入\texinline{intersections}库来实现,请先看一个简单的示例:
\begin{texlst}
	\usetikzlibrary{intersections}
	\begin{tikzpicture}[every node/.style={opacity=1, black, above left}]
		\draw [help lines] grid (3, 2);
		\draw [name path=ellipse] (2,0.5) ellipse(0.75cm and 1cm);
		\draw [name path=rectangle, rotate=10] (0.5, 0.5) rectangle +(2,1);
		\fill [red, opacity=0.5, name intersections={of=ellipse and rectangle}]
		(intersection-1) circle (2pt) node {1}
		(intersection-2) circle (2pt) node {2};
	\end{tikzpicture}
\end{texlst}

这里默认的名字就是\texinline{intersection-数字},我们也可以自己设置交点的名称。另外还可以交点的总数设置给变量。来看下面一个例子:
\begin{texlst}
	\usetikzlibrary{intersections}
	\begin{tikzpicture}
		\clip (-2, -2) rectangle (2, 2);
		\draw[name path=curve 1] (-2,-1) .. controls (8,-1) and (-8,1) .. (2,1);
		\draw[name path=curve 2] (-1,-2) .. controls (-1,8) and (1, -8) .. (1,2);
		\fill [name intersections={of=curve 1 and curve 2, name=i, total=\t}]
		[red, opacity=0.5, every node/.style={ above left, black, opacity=1 }]
		\foreach \s in {1,...,\t} {(i-\s) circle (2pt) node {\footnotesize\s}};
	\end{tikzpicture}
\end{texlst}

还可以用\texinline{by}来给交点命名。
\begin{texlst}
	\usetikzlibrary{intersections}
	\begin{tikzpicture}
		\clip (-2, -2) rectangle (2, 2);
		\draw[name path=curve 1] (-2,-1) .. controls (8,-1) and (-8,1) .. (2,1);
		\draw[name path=curve 2] (-1,-2) .. controls (-1,8) and (1, -8) .. (1,2);
		\fill [name intersections={of=curve 1 and curve 2, by={a, b}}]
		[red, opacity=0.5, every node/.style={ above left, black, opacity=1 }]
		(a) circle (2pt)
		(b) circle (2pt);
	\end{tikzpicture}
\end{texlst}

甚至还可以用\texinline{foreach}的语法来给交点加上\texinline{lable}。
\begin{texlst}
	\usetikzlibrary{intersections}
	\begin{tikzpicture}
		\clip (-2, -2) rectangle (2, 2);
		\draw[name path=curve 1] (-2,-1) .. controls (8,-1) and (-8,1) .. (2,1);
		\draw[name path=curve 2] (-1,-2) .. controls (-1,8) and (1, -8) .. (1,2);
		\fill [name intersections={of=curve 1 and curve 2, by={
							[label=above right:a],[label=above right:...], [label=above right:i]}}];
	\end{tikzpicture}
\end{texlst}

用\texinline{sort by}来根据某条路径来排列交点的顺序。
\begin{texlst}
	\usetikzlibrary{intersections}
	\begin{tikzpicture}
		\clip (-2, -2) rectangle (2, 2);
		\draw[name path=curve 1] (-2,-1) .. controls (8,-1) and (-8,1) .. (2,1);
		\draw[name path=curve 2] (-1,-2) .. controls (-1,8) and (1, -8) .. (1,2);
		\fill [name intersections={of=curve 1 and curve 2, sort by=curve 1, by={
							[label=above right:a],[label=above right:...], [label=above right:i]}}];
	\end{tikzpicture}
\end{texlst}

\begin{texlst}
	\usetikzlibrary{intersections}
	\begin{tikzpicture}
		\clip (-2, -2) rectangle (2, 2);
		\draw[name path=curve 1] (-2,-1) .. controls (8,-1) and (-8,1) .. (2,1);
		\draw[name path=curve 2] (-1,-2) .. controls (-1,8) and (1, -8) .. (1,2);
		\fill [name intersections={of=curve 1 and curve 2, sort by=curve 2, by={
							[label=above right:a],[label=above right:...], [label=above right:i]}}];
	\end{tikzpicture}
\end{texlst}

\section{相对坐标}
\texinline{+}和\texinline{++}是两个相对坐标操作符,区别在于\texinline{+}不改变当前的基点,而\texinline{++}操作后会将自身变为当前的基点。

旋转相对操作符\texinline{turn}
\begin{texlst}
	\begin{tikzpicture}
		\draw (0,0) -- (1,1) -- ([turn]-45:1cm) -- ([turn]-30:1cm);
	\end{tikzpicture}
\end{texlst}

\begin{texlst}
	\begin{tikzpicture}[delta angle=30, radius=1cm]
		\draw (0,0) arc [start angle=0] -- ([turn]0:1cm)
		arc [start angle=30] -- ([turn]0:1cm)
		arc [start angle=60] -- ([turn]30:1cm);
	\end{tikzpicture}
\end{texlst}

\texinline{turn}操作符还能接着\texinline{bend left/right}的角度继续变化。
\begin{texlst}
	\tikz \draw (0,0) to [bend left] (2,1) -- ([turn]0:1cm);
\end{texlst}
还可以接着\texinline{plots}后面。
\begin{texlst}
	\tikz \draw plot coordinates {(0,0) (1,1) (2,0) (3,0)} -- ([turn]30:1cm);
\end{texlst}

如果\texinline{turn}后面接的不是极坐标,比如\texinline{([turn]1,1)},这个效果相当于一个长度为$\sqrt2$偏角为$45^\circ$。

一般情况下大括号对对于相对坐标的计算是没有效果的,如下例:
\begin{texlst}
	\begin{tikzpicture}
		\draw (0,0) -- ++(1,0) -- ++(0,1) -- ++(-1,0);
		\draw[red] (2,0) -- ++(1,0) {-- ++(0,1)} -- ++(-1,0);
	\end{tikzpicture}
\end{texlst}

仅当设置\texinline{current point is local}时会将之视为局部相对坐标计算,如下:
\begin{texlst}
	\begin{tikzpicture}
		\draw (0,0) -- ++(1,0) -- ++(0,1) -- ++(-1,0);
		\draw[red] (2,0) -- ++(1,0) {[current point is local]-- ++(0,1)} -- ++(-1,0);
	\end{tikzpicture}
\end{texlst}

\section{坐标计算}
\subsection{坐标的四则运算}
\tikzname\ 中的坐标计算需要加载\texinline{calc}库。语法为:\texinline{([options]$coordinate computation$)}
\begin{texlst}
	\usetikzlibrary{calc}
	\begin{tikzpicture}
		\draw[help lines] (0,0) grid (3,2);

		\fill [red] ($2*(1,1)$) circle (2pt);
		\fill [green] (${1+1}*(1,.5)$) circle (2pt);
		\fill [blue] ($cos(0)*sin(90)*(1,1)$) circle (2pt);
		\fill [black] (${3*(4-3)}*(1,0.5)$) circle (2pt);
	\end{tikzpicture}
\end{texlst}

\subsection{比例修饰符}
\tikzname\ 还支持比例修饰符,语法为:\texinline{($(point1)!number!angle:(point2)$)}
\begin{texlst}
	\usetikzlibrary{calc}
	\begin{tikzpicture}
		\draw [help lines] (0,0) grid (3,2);

		\draw (1,0) -- (3,2);

		\foreach \i in {0, 0.2, 0.5, 0.9, 1}
		\node at ($(1,0)!\i!(3,2)$) {\i};
	\end{tikzpicture}
\end{texlst}

再看一个有角度的例子:
\begin{texlst}
	\usetikzlibrary{calc}
	\begin{tikzpicture}
		\draw [help lines] (0,0) grid (3,3);

		\coordinate (a) at (1,0);
		\coordinate (b) at (3,2);

		\draw[->] (a) -- (b);

		\coordinate (c) at ($(a)!1!10:(b)$);

		\draw[->, red] (a) -- (c);

		\fill ($(a)!0.5!10:(b)$) circle (2pt);
	\end{tikzpicture}
\end{texlst}

再来看一个有点复杂的例子:
\begin{texlst}
	\usetikzlibrary{calc}
	\begin{tikzpicture}
		\draw [help lines] grid (4,4);

		\foreach \i in {0, 0.125, ..., 2}
		\fill ($(2,2)!\i!\i*180:(3,2)$) circle (2pt);
	\end{tikzpicture}
\end{texlst}

比例修饰算法可以接力完成。
\begin{texlst}
	\usetikzlibrary{calc}
	\begin{tikzpicture}
		\draw [help lines] grid (3,2);

		\draw (0,0) -- (3,2);
		\draw[red] ($(0,0)!.3!(3,2)$) -- (3,0);
		\fill[red] ($(0,0)!.3!(3,2)!.7!(3,0)$) circle (2pt);
	\end{tikzpicture}
\end{texlst}

\subsection{距离修饰符}
坐标计算还有一种距离修饰符,语法为\texinline{($(point1)!dimension!angle:(point2)$)}。
\begin{texlst}
	\usetikzlibrary{calc}
	\begin{tikzpicture}
		\draw [help lines] (0,0) grid (3,2);

		\draw (1,0) -- (3,2);

		\foreach \i in {0cm, 1cm, 15mm}
		\node at ($(1,0)!\i!(3,2)$) {\i};
	\end{tikzpicture}
\end{texlst}

再来看一个带角度的例子:
\begin{texlst}
	\usetikzlibrary{calc}
	\begin{tikzpicture}
		\draw [help lines] (0,0) grid (3,2);

		\coordinate (a) at (1,0);
		\coordinate (b) at (3,1);

		\draw (a) -- (b);

		\coordinate (c) at ($(a)!0.25!(b)$);
		\coordinate (d) at ($(c)!1cm!90:(b)$);

		\draw[<->] (c) -- (d) node[sloped, midway, above] {1cm};
	\end{tikzpicture}
\end{texlst}

\subsection{投影修饰符}
投影修饰符用来求中间点在前后两点连线上的投射点,语法为\texinline{($(p1)!(p2)!angle:(p3)$)}。
\begin{texlst}
	\usetikzlibrary{calc}
	\begin{tikzpicture}
		\draw [help lines] (0,0) grid (3,2);

		\coordinate (a) at (0,1);
		\coordinate (b) at (3,2);
		\coordinate (c) at (2.5,0);

		\draw (a) -- (b) -- (c) -- cycle;

		\draw[red] (a) -- ($(b)!(a)!(c)$);
		\draw[orange] (b) -- ($(a)!(b)!(c)$);
		\draw[blue] (c) -- ($(b)!(c)!(a)$);
	\end{tikzpicture}
\end{texlst}

\chapter{路径构建}
路径包括构建路径,以及路径操作。本章主要讲解构建这一部分。\texinline{\path (0,0) --
	(1,1);}会新建一条路径,但是表面上看不到什么变化。路径的操作要将操作的内容放置在\texinline{[]}中,比如\texinline{\path[draw]
	(0,0) -- (1,1);},而我们平时用的\texinline{\draw}命令其实只是\texinline{\path[draw]}的快捷方式。
\section{移动 Move-To}
这个操作就像是提着笔移动到一个位置,但是没有下笔画任何内容。
\begin{texlst}
	\begin{tikzpicture}
		\draw (0,0) -- (2,0) (0,1) -- (2,1);
	\end{tikzpicture}
\end{texlst}
上面的例子中先是移动到\texinline{(0,0)},然后下笔画到\texinline{(2,0)};接下来移动到\texinline{(0,1)},再下笔画到\texinline{(2,1)}。
其中的\texinline{--(2,0)}和\texinline{--(2,1)}这个操作称为画线操作(line-to)。

有一个特殊的坐标称之为\texinline{current subpath start}, 这个值永远是最后一个移动操作的坐标。如下例:
\begin{texlst}
	\tikz[line width=2mm]
	\draw (0,0) -- (1,0) -- (1,1) -- (0,1) -- (current subpath start);
\end{texlst}

\section{直线连接符\tt{--}}
这个是最常用的,与此操作符类似的还有\texinline{to}和\texinline{edge},\texinline{--}是最简单的,无法搭配选项使用。

注意,在需要图形实现闭包的情况下,最后需要用\texinline{cycle}来代替初始点。否则接头的地方实现不是完美的接合。

\section{直角连接符\tt{|-}\quad、\tt{-|}}
这两个操作符使用起来非常直观。
\begin{texlst}
	\begin{tikzpicture}
		\draw (0,0) node (a) [draw] {A} (1,1) node(b) [draw] {B};

		\draw (a.north) |- (b.west);
		\draw [color=red] (a.east) -| (2,1.5) -| (b.north);
	\end{tikzpicture}
\end{texlst}

\section{曲线连接符 .. controls a and b ..}
曲线连接所控制的两个点分别与起点和终点正切,如果只提供一个控制点,则认为第二个控制点与第一个相同。
\begin{texlst}
	\begin{tikzpicture}[scale=0.9]
		\draw[line width=10pt] (0,0) .. controls (1,1) .. (4,0) .. controls (5,0) and (5,1) .. (4,1);
		\draw[color=gray] (0,0) -- (1,1) -- (4,0) -- (5,0) -- (5,1) -- (4,1);
	\end{tikzpicture}
\end{texlst}

\section{矩形 rectangle}
绘制矩形比较简单,只需要提供两个对角的点坐标即可。
\begin{texlst}
	\begin{tikzpicture}
		\foreach \a in {0,1,...,360}
		\draw [rotate=\a] (-2,-1) rectangle (2,1);
	\end{tikzpicture}
\end{texlst}
\begin{tikzpicture}
	\foreach \a in {0,3,...,360}
	\draw [rotate=\a] (-2,-1) rectangle (2,1);
\end{tikzpicture}
\begin{tikzpicture}
	\foreach \a in {0,7,...,360}
	\draw [rotate=\a] (-2,-1) rectangle (2,1);
\end{tikzpicture}
\begin{tikzpicture}
	\foreach \a in {0,13,...,360}
	\draw [rotate=\a] (-2,-1) rectangle (2,1);
\end{tikzpicture}

以上三个图形是分别按3、7、13的角度偏转矩形时产生的。

\section{圆角 rounded corners}
\texinline{rounded corners}命令的默认值是\texinline{4pt},角的形状默认是\texinline{sharp corners}。

\section{圆形和椭圆 circle ellipse}
这两个命令是一样的,不过我还是习惯画圆的时候用\texinline{circle},而画椭圆的时候用\texinline{ellipse}。
\begin{texlst}
	\begin{tikzpicture}[every circle/.style={fill}]
		\draw [x radius=1, y radius=2, rotate=45] circle;
		\draw (2,0) ellipse [radius=1];
		\draw (1,1) circle (0.5);
	\end{tikzpicture}
\end{texlst}
可以看到\texinline{every circle/.style}是不起作用的,我查看了一个,\tikzname 其实提供了\\
\texinline{every circle node/.style},当然,其中的\texinline{circle}也可以是其他的形状。

\section{圆弧 arc}
圆弧一般都会提供\texinline{start angle}和\texinline{end angle},另外还有一个\texinline{delta angle}。如果\texinline{end
	angle}为空,则用\texinline{start angle}加上\texinline{delta angle}来代替。如果\texinline{start angle}为空,则用\texinline{end
	angle}减去\texinline{delta angle}来代替。若三者都提供了\texinline{delta angle}不起任何作用。
\begin{texlst}
	\begin{tikzpicture}[radius=1cm, delta angle=30]
		\draw (-1,0) -- +(3.5,0);
		\draw (1,0) ++(210:2cm) -- +(30:4cm);
		\draw (1,0) +(0:1cm) arc [start angle=0];
		\draw (1,0) +(180:1cm) arc [start angle=180];
		\path (1,0) ++(15:.75cm) node{$\alpha$};
		\path (1,0) ++(15:-.75cm) node{$\beta$};
	\end{tikzpicture}
\end{texlst}
\emph{注意:}不能用\texinline{\node (1,0)+(0:1cm) {text};},而要使用上面代码段中的\texinline{\path}语法。

\section{网格grid}
\texinline{grid}一般用来画参照线,对应的有\texinline{help
	lines}的预设样式,另外还可以用\texinline{step}\texinline{xstep}\texinline{ystep}来设置步长。

\section{抛物线 parabola}
这个我目前还没有用到过,先占个坑,看以后是不是会有需求。
\begin{texlst}
	\begin{tikzpicture}
		\draw (0,0) rectangle (1,1.5)
		(0,0) parabola (1,1.5);
		\draw[xshift=1.5cm] (0,0) rectangle (1,1.5)
		(0,0) parabola[bend at end] (1,1.5);
		\draw[xshift=3cm] (0,0) rectangle (1,1.5)
		(0,0) parabola bend (.75, 1.75) (1,1.5);
	\end{tikzpicture}
\end{texlst}

另外还有\texinline{bend pos}来设置拐点的比例位置,另外还有\texinline{parabola height}用来设置抛物线的高度。
\begin{texlst}
	\begin{tikzpicture}
		\draw [help lines] (0,0) grid (3,2);
		\draw (-1,0) parabola[bend pos=0.4] bend +(0,2) +(3,0);
	\end{tikzpicture}
\end{texlst}

\section{正弦和余弦曲线 Sin Cos}
两点之间的连接曲线是$[0, \pi/2]$的正弦曲线,会根据距离缩放和偏移。

\section{SVG}
这个没有学过,就不往里面跳了。

\section{Plot}
这个会在后面专门讲。

\section{To 操作符}
这个在讲直线连接符的时候提到过,与\texinline{--}只能绘制直线不一样,\texinline{to}操作符可以带很多选项来实现路径形状的变化。

在\texinline{to}构建的路径形成之前,\texinline{\tikztostart}、\texinline{\tikztotarget}和\texinline{\tikztonodes}会被设定,内容从名字上就可以推断出来了。

有了这几个宏就可以实现一些方便的操作,比如:
\begin{texlst}
	\begin{tikzpicture}[skip loop/.style={to path={-- +(0,0.5) -| (\tikztotarget) \tikztonodes}, -Stealth,shorten >=1pt,
					draw=black!50, thick, rounded corners}]
		\draw (-1,0) -- (3,0);
		\draw[skip loop] (0,0) to (2,0);
	\end{tikzpicture}
\end{texlst}

\begin{texlst}
	\begin{tikzpicture}
		\draw (0,0) to node [sloped, above] {x} (3,2);
		\draw (0,0) to [out=90, in=180] node [sloped, above] {x} (3,2);
		\draw (0,0) to [bend right=45] node [sloped, above] {y} (3,2);
		\draw (0,0) to [bend right=45, edge label'=y'] (3,2);
	\end{tikzpicture}
\end{texlst}

一些具体的路径变化命令,比如\texinline{bend
	right}等这一节里面都没有讲到,这一章主要是讲路径的构造方式,所以细节这里暂时略过。
另外还有一个\texinline{every to}可以设置全局的\texinline{to}样式。

\section{循环命令foreach}
这个命令我比较熟练了,之前在讲矩形路径构造的时候已经动手写过了。

\section{Let赋值命令}
\texinline{\let}命令需要加载\texinline{calc}库。用一个示例来记录一下用法吧。
\begin{texlst}
	\begin{tikzpicture}
		\draw [help lines] (0,0) grid (3,3);

		\coordinate (a) at (rnd, rnd);
		\coordinate (b) at (3-rnd, 3-rnd);
		\draw (a) -- (b);

		\node (c) at (1,2) {x};

		\draw let \p1 = ($(a)!(c)!(b) - (c)$),
		\n1 = {veclen(\x1, \y1)}
		in circle [at=(c), radius=\n1];
	\end{tikzpicture}
\end{texlst}
其中\texinline{\p}为点寄存器, \texinline{\n}为数值寄存器,\texinline{\x}、\texinline{\y}分别为$x$和$y$坐标寄存器。

\section{保存和使用路径Save Path \&\ Use Path}
\begin{texlst}
	\usetikzlibrary{intersections}
	\begin{tikzpicture}
		\path[save path=\pathA, name path=A] (0,1) to [bend left] (1,0);
		\path[save path=\pathB, name path=B] (0,0) .. controls (.33, .1)
		and (.66, .9) .. (1,1);

		\fill [name intersections={of=A and B, by=p}] (p) circle (1pt);

		\draw[blue] [use path=\pathA];
		\draw[red] [use path=\pathB];

	\end{tikzpicture}

\end{texlst}

\chapter{操纵路径的命令}
构建路径后,可以对路径进行多种操纵以改变路径的呈现形态。一般常用的命令以以下这些:

\noindent
\begin{ditemize}
	\forcsvlist{\tcbitem}{draw, fill, filldraw, pattern, shade, shadedraw, clip, useasboundingbox}

\end{ditemize}

\section{颜色color}
可以用\texinline{\colorlet}命令来定义新的颜色,\texinline{black!50}会产生中灰色\tikz\filldraw[black!50] circle (5pt);
。这样的颜色取值模式很方便,并且可以与其他颜色继续混合。

\section{绘制draw}
\texinline{\draw}命令用来绘制路径,除了路径的颜色以外,还有一些其他的属性会影响整个图形的外观。
\subsection{线条粗细、端头、接头}
\texinline{line width}属性有以下一些预设值:
\begin{ditemize}
	\forcsvlist{\tcbitem}{ultra thin 0.1pt, very thin 0.2pt,thin 0.4pt, semithick 0.6pt, thick 0.8pt, very thick 1.2pt, ultra thick 1.6pt}
\end{ditemize}
\texinline{line cap}有三种样式\forcsvlist{\texinline}{round, butt, rect}。
\begin{texlst}
	\begin{tikzpicture}
		\begin{scope}[line width=10pt]
			\draw[line cap=round] (0,1) -- +(1,0);
			\draw[line cap=butt] (0,.5) -- +(1,0);
			\draw[line cap=rect] (0,0) -- +(1,0);
		\end{scope}
		\draw[white, line width=2pt] (0,0) -- +(1,0) (0,.5) -- +(1,0) (0,1) -- +(1,0);
	\end{tikzpicture}
\end{texlst}

转角处的连接也有三种样式\forcsvlist{\texinline}{round, bevel, miter}
\begin{texlst}
	\begin{tikzpicture}[line width=10pt]
		\draw[line join=round] (0,0) -- ++(.5,1) -- ++(.5,-1);
		\draw[line join=bevel] (1.25,0) -- ++(.5,1) -- ++(.5,-1);
		\draw[line join=miter] (2.5,0) -- ++(.5,1) -- ++(.5,-1);
		\useasboundingbox(0,1.5);
	\end{tikzpicture}
\end{texlst}

\texinline{miter}尖角模式下,如果夹角太小,会导致尖头太长,这里可以通过设置\texinline{miter
	limit},如果尖头的长度超过线宽的\texinline{miter limit}倍就自动转换成平头的,默认值是10。
\begin{texlst}
	\begin{tikzpicture}[line width=5pt, scale=0.9]
		\draw (0,0) -- ++(5,.5) -- ++(-5, .5);
		\draw[miter limit=25] (0,-2) -- ++(5,.5) -- ++(-5, .5);
	\end{tikzpicture}
\end{texlst}

\subsection{点划线模式 dash pattern}
\begin{texlst}
	\begin{tikzpicture}[dash pattern=on 2pt off 3pt on 4pt off 4pt]
		\draw (0,0) -- (3.5cm, 0);
	\end{tikzpicture}
\end{texlst}

此外还可以通过\texinline{dash phase}来错开。
\begin{texlst}
	\begin{tikzpicture}[dash pattern=on 20pt off 10pt]
		\draw[dash phase=0pt] (0pt, 3pt) -- (3.5cm, 3pt);
		\draw[dash phase=10pt] (0pt, 0pt) -- (3.5cm, 0pt);
	\end{tikzpicture}
\end{texlst}

还可以将\texinline{dash pattern}与\texinline{dash phase}写到一起。
\begin{texlst}
	\begin{tikzpicture}
		\draw[dash=on 20pt off 10pt phase 0pt] (0pt,3pt) -- (3.5cm, 3pt);
		\draw[dash=on 20pt off 10pt phase 10pt] (0pt,0pt) -- (3.5cm, 0pt);
	\end{tikzpicture}
\end{texlst}
注意,这里\texinline{pattern}给省略了。

其他的所有线的样式其实都可以从\texinline{dash pattern}里面变化而来,比如实线(solid)就是\\
\texinline{on 1pt off 0pt},等等。\tikzname 提供了一些预设好的样式:
\begin{ditemize}
	\forcsvlist{\tcbitem}{solid, dotted, densly dotted, loosely dotted, dashed, loosely dash, densly dashed, dash dot,
		densely dash dot, loosely dash dot, dash dot dot, densely dash dot dot, loosely dash dot dot}
\end{ditemize}

\subsection{双线 double}
双线的中间默认是白色的,初始线距0.6pt。
\begin{texlst}
	\begin{tikzpicture}
		\draw[double=red!10, red!50, double distance=5pt] plot[smooth cycle] coordinates{(0,0) (1,1) (1,0) (0,1)};
	\end{tikzpicture}
\end{texlst}
此外,线距还有两个变种,线中心距\texinline{double distance between line center}和等号中心距\\\texinline{double equal sign
	distance}。

\section{箭头 arrow tips}
普通的箭头经常使用,其他的箭头样式后面再看。

\section{填充 fill}
填充颜色只需要给\texinline{fill}选项指定一个颜色就行。此外,\tikzname 还提供了一些预设的图案\texinline{pattern},
需要加载\texinline{patterns}库。
\begin{texlst}
	\begin{tikzpicture}
		\draw[pattern=dots] (0,0) circle (1cm);
		\draw[pattern=fivepointed stars, pattern color=red] (0,0) rectangle (3,1);
	\end{tikzpicture}
\end{texlst}

填充模式默认为\texinline{nonzero rule},就是全部填充。另外一种叫\texinline{even odd
	rule}也就是根据虚拟的一条线穿过路径是偶数的情况下认为是在外部,奇数为内部。

\begin{texlst}
	\begin{tikzpicture}
		\draw[even odd rule, fill=green!30, draw=red!30]
		\foreach \a in {0, 5, ..., 360}{(\a:1.5cm) circle (.8cm)};
	\end{tikzpicture}
\end{texlst}

\section{图片填充}
先来看一个示例:
\begin{texlst}
	\begin{tikzpicture}
		\draw [help lines] (0,0) grid (3,2);
		\filldraw [fill=blue!10, draw=blue, thick] (1.5,1) circle (1)
		[path picture={
						\node at (path picture bounding box.center){
							This is a long text.
						};
					}];
	\end{tikzpicture}
\end{texlst}

再来一个:
\begin{texlst}
	\begin{tikzpicture}[cross/.style={path picture={
							\draw[black]
							(path picture bounding box.south east) --
							(path picture bounding box.north west)
							(path picture bounding box.south west) --
							(path picture bounding box.north east);
						}}]
		\draw [help lines] (0,0) grid (3,2);
		\filldraw [cross, fill=blue!10, draw=blue, thick] (1,1) circle (1);
		\path [cross, top color=red, draw=red, thick] (2,0) -- (3,2) -- (3,0);
	\end{tikzpicture}
\end{texlst}

第二个红色的示例我目前还没有搞清楚,后来有机会再深入吧。这里当然也可以用磁盘上的图片,就不试验了。

\section{渐变 Shading}
渐变与填充相比,区别就是填充的是纯色,而渐变填充的颜色是有变化的。

\begin{texlst}
	\begin{tikzpicture}
		\shade circle (1cm);
	\end{tikzpicture}
\end{texlst}
默认的渐变是轴向的\texinline{axis},上部的颜色是灰,下部是白。另外还有两种\forcsvlist{\texinline}{radial 辐射, ball
	球形}。

\begin{texlst}
	\begin{tikzpicture}
		\shadedraw [shading=axis, shading angle=135, lower left=blue] (0,0) rectangle (1,1);
		% 这个里面的渐变角度和指定的lower left的颜色位置是有叠加效果的。
		\shadedraw [xshift=1.5cm, shading=radial, inner color=blue!50, outer color=green!20] (0,0) rectangle (1,1);
		\shadedraw [xshift=3.5cm, shading=ball, ball color=red!60, draw=red] (0,0.5) circle(0.5cm);
	\end{tikzpicture}
\end{texlst}

其他更多的渐变用法后面会继续学到。

\section{包围盒子 bounding box}
\texinline{use as bounding box}这个功能很强大,这个让一些设置变得非常简单。
\begin{texlst}
	Left of picture
	\begin{tikzpicture}
		\draw[use as bounding box] (2,0) rectangle (3,1);
		\draw (1,0) -- (4,.75);
	\end{tikzpicture}
	right of picture
\end{texlst}

另外,还有一个\texinline{current bounding box}也非常好用。
\begin{texlst}
	\begin{tikzpicture}
		\draw [red] (0,0) circle (2pt);
		\draw [red] (2,1) circle (3pt);

		\draw (current bounding box.south west) rectangle (current bounding box.north east);

		\draw [red] (3,-1) circle (4pt);

		\draw[thick] (current bounding box.south west) rectangle (current bounding box.north east);
	\end{tikzpicture}
\end{texlst}

\texinline{trim left}和\texinline{trim right}会将设定值的左边或者右边忽略掉。如果设定值是一个坐标的话只会使用$x$的值。

\begin{texlst}
	text before image
	\begin{tikzpicture}[trim left] %默认值是0pt
		\draw (-1,-1) grid (1,2);
		\fill (0,0) circle (5pt);
	\end{tikzpicture}
	text after image.
\end{texlst}

\section{裁剪 Clip}
\begin{texlst}
	\begin{tikzpicture}
		\clip (0,0) circle (1cm);
		\fill[red] (1,0) circle (1cm);
	\end{tikzpicture}
\end{texlst}

\section{多重动作 Preaction \& Postaction}
\texinline{preaction}和\texinline{postaction}分别可以在绘制路径之前与之后进行额外的动作。
\begin{texlst}
	\begin{tikzpicture}
		\draw [help lines] (0,0) grid (3,2);

		\draw[preaction={draw, line width=4mm, blue}]
		[line width=2mm, red] (0,0) rectangle (2,2);
	\end{tikzpicture}
\end{texlst}

用这个还可以做一个阴影效果:
\begin{texlst}
	\begin{tikzpicture}
		\draw [help lines] (0,0) grid (3,2);
		\draw [preaction={fill=black, opacity=.5, transform canvas={xshift=1mm, yshift=-1mm}}]
		[fill=red] (0,0) rectangle (1,2)
		(1,2) circle (5mm);
	\end{tikzpicture}
\end{texlst}

\texinline{preaction}或者\texinline{postaction}的效果还可以叠加。
\begin{texlst}
	\usetikzlibrary{patterns}
	\begin{tikzpicture}
		\draw [help lines] (0,0) grid (3,2);
		\draw[pattern=fivepointed stars] [preaction={fill=black, opacity=.5, transform canvas={xshift=1mm, yshift=-1mm}}]
		[preaction={top color=blue, bottom color=white}]
		(0,0) rectangle (1,2)
		(1,2) circle (5mm);
	\end{tikzpicture}
\end{texlst}

下面是一个复杂的示例:
\begin{texlst}
	\usetikzlibrary{fadings, patterns}
	\begin{tikzpicture}
		[button/.style={
		preaction={fill=black, path fading=circle with fuzzy edge 20 percent, opacity=0.5, transform
		canvas={xshift=1mm, yshift=-1mm}},
		preaction={pattern=#1, path fading=circle with fuzzy edge 15 percent},
		preaction={top color=white,
				bottom color=black!50,
				shading angle=45,
				path fading=circle with fuzzy edge 15 percent, opacity=.2},
		preaction={path fading=fuzzy ring 15 percent,
				top color=black!5,
				bottom color=black!80,
				shading angle=45},
		inner sep=2ex
		},
		button/.default=horizontal lines light blue,
		circle
		]
		\draw [help lines] (0,0) grid (4,3);

		\node[button] at (2.2,1) {\Huge Big};
		\node[button=crosshatch dots light steel blue, text=white] at (1,1.5){Small};
	\end{tikzpicture}
\end{texlst}

这个示例对我来讲有点复杂,因为\texinline{patterns}里面的内容了解很少。以后再说。

下面再看两个\texinline{postaction}的示例:
\begin{texlst}
	\begin{tikzpicture}
		\draw [help lines] (0,0) grid (3,2);

		\draw
		[postaction={draw, line width=2mm, blue}]
		[line width=4mm, red, fill=white] (0,0) rectangle (2,2);
	\end{tikzpicture}
\end{texlst}

\begin{texlst}
	\begin{tikzpicture}
		\draw[help lines] (0,0) grid (3,2);
		\draw
		[postaction={path fading=south, fill=white}]
		[postaction={path fading=south, fading angle=45, fill=blue, opacity=.5}]
		[left color=black, right color=red, draw=white, line width=2mm]
		(0,0) rectangle (1, 2)
		(1,2) circle (5mm);
	\end{tikzpicture}
\end{texlst}

\section{装饰和变形 Decorating and Morphing}
装饰给路径添加一些内容,而变形则是改变路径的形状。

\begin{texlst}
	\usetikzlibrary{decorations.pathmorphing}
	\begin{tikzpicture}
		\draw (0,0) rectangle (3,2);
		\draw [red, decorate, decoration=zigzag] (0,0) rectangle (3,2);
	\end{tikzpicture}
\end{texlst}

\begin{texlst}
	\begin{tikzpicture}
		\node [circular drop shadow={shadow scale=1.05},
			minimum size=3.13cm,
			decorate,
			decoration=zigzag,
			fill=blue!20,
			draw,
			thick,
			circle] {Hello!};
	\end{tikzpicture}
\end{texlst}

具体的内容会在后面的章节里面详细了解。

\chapter{arrows箭头}
箭头除了最普通的那种以外,其他的需要加载\texinline{arrows.meta}库,以前的\texinline{arrows}\texinline{arrows.space}已经过时。另外新的功能强大的箭头的命名规则是首字母大写。
\begin{texlst}
	\begin{tikzpicture}
		\draw[{stealth[green!50!black]}-{Stealth[red]Stealth[blue]>Latex}] (0,0) -- (1,0);
	\end{tikzpicture}
\end{texlst}
路径上可以有多个箭头。

\section{箭头显示的规则}
\texinline{tips}可以设置了\forcsvlist{\texinline}{true, proper, on draw, on proper draw, never or
	false},以下一些示例可以说明显示箭头与否的规则:
\begin{texlst}
	\begin{tikzpicture}
		\matrix[row sep=1em, column sep=1em]{
			\node[circle, draw]{1}; & \draw [<->];                                           \\ %没有路径,所以没有箭头
			\node[circle, draw]{2}; & \draw [<->] (0,0);                                     \\ %因为没有路径,所以箭头退化
			\node[circle, draw]{3}; & \draw [<->, tips=proper] (0,0);                        \\ %proper选项抑制箭头显示
			\node[circle, draw]{4}; & \draw [<->] (0,0) -- (1,0);                            \\ %正常情况
			\node[circle, draw]{5}; & \draw [<->] (0,0) -- (1,0) (2, 0) -- (3, 0);           \\ %有两个子路径,只有最后一个有箭头
			\node[circle, draw]{6}; & \draw [<->] (0,0) -- (1,0) (2,0);                      \\%第二个子路径退化,所以箭头也退化
			\node[circle, draw]{7}; & \draw [<->, tips=on proper draw] (0,0) -- (1,0) (2,0); \\%第二个子路径退化,但是on proper draw抑制箭头显示
			\node[circle, draw]{8}; & \draw [<->] (0,0) circle[radius=2pt] (2,0) -- (3,0);   \\%有两个子路径,但是其中一个是封闭的,所以不显示任何箭头
		};
	\end{tikzpicture}
\end{texlst}

\section{箭头的外观}
\subsection{尺寸}
\texinline{length}选项可以设置箭头的长度,如:
\begin{texlst}
	\begin{tikzpicture}
		\draw [-{Stealth[length=5mm]}] (0,0) -- (2,0);
		\draw [|<->|] (1.5,0.4) -- node[above=1mm]{5mm} (2,0.4);
	\end{tikzpicture}
\end{texlst}

还有一种标识方法,\texinline{length=3pt 5},其中5是倍数,基数是标准线的宽度$0.4pt$,$0.4pt \times 5 + 3pt = 5pt$。

另外,在双线模式下才生效,写做\texinline{length=0pt 3 0}样式,其中的第三个参数称为线宽因子(Line Width
Factors)。其计算公式如下:
\begin{align*}
	\omega   & = \langle outer factor\rangle\omega_0 + (1-\langle outer factor\rangle )\omega_t \\
	\omega_0 & 双线的线宽                                                                            \\
	\omega_t & 双线的总线宽
\end{align*}

下面来看三个示例:
\begin{texlst}
	\begin{tikzpicture}
		\draw [line width=1pt, double distance=3pt, arrows={-Latex[length=0pt 3 0]}] (0,0) -- (1,0);
		\draw[|<->|] (1cm-15pt, 0.4) --node[above] {15mm} (1,0.4);
		% 1pt * 0 + (1-0) * 5 = 5 * 3 = 15pt
		\draw [yshift=-1cm, line width=1pt, double distance=3pt, arrows={-Latex[length=0pt 3 0.5]}] (0,0) -- (1,0);
		% 1pt * 0.5 + (1-0.5) * 5 = 3 * 3 = 9pt
		\draw[|<->|, yshift=-1cm] (1cm-9pt, 0.4) --node[above] {9mm} (1,0.4);
		\draw [yshift=-2cm, line width=1pt, double distance=3pt, arrows={-Latex[length=0pt 3 1]}] (0,0) -- (1,0);
		% 1pt * 1 + (1-1) * 5 = 1 * 3 = 3pt
		\draw[|<->|, yshift=-2cm] (1cm-3pt, 0.4) --node[above] {3mm} (1,0.4);
	\end{tikzpicture}
\end{texlst}

\texinline{length}和\texinline{width}可以分别设置箭头的长度和宽度。另外还有一个\texinline{width'}选项,设置的格式为\\
\texinline{width'=0pt .5}, 这个写法的意思是宽度是其长度的一半。

\begin{texlst}
	\begin{tikzpicture}
		\draw [arrows={-Stealth[length=10pt, inset=5pt]}] (0,0) -- (1,0);
		\draw [arrows={-Stealth[length=10pt, inset=2pt]}, yshift=-1cm] (0,0) -- (1,0);
	\end{tikzpicture}
\end{texlst}

另外,\texinline{inset'}选项跟\texinline{width'}一样,可以设置\texinline{inset}与长度的倍数关系。

\texinline{angle}用来设置箭头的角度,并且角度后面也可以附带设置箭头的长度。如下例:
\begin{texlst}
	\begin{tikzpicture}
		\draw [arrows={-Stealth[inset=0pt, angle=90:10pt]}] (0,0) -- (1,0);
		\draw [yshift=-1cm, arrows={-Stealth[inset=0pt, angle=30:20pt]}] (0,0) -- (1,0);
	\end{tikzpicture}
\end{texlst}

\subsection{缩放}
用\texinline{scale}选项可以设置缩放比例,如下例:
\begin{texlst}
	\begin{tikzpicture}
		\draw [arrows={-Stealth[]}] (0,1) -- (1,1);
		\draw [arrows={-Stealth[scale=1.5]}, yshift=-1cm] (0,1) -- (1,1);
		\draw [arrows={-Stealth[scale=2]}, yshift=-2cm] (0,1) -- (1,1);
	\end{tikzpicture}
\end{texlst}

另外还有长度和宽度分别缩放的选项\texinline{length scale}和\texinline{width scale}:
\begin{texlst}
	\begin{tikzpicture}
		\draw [arrows={-Stealth[]}] (0,1) -- (1,1);
		\draw [arrows={-Stealth[scale length=1.5]}, yshift=-1cm] (0,1) -- (1,1);
		\draw [arrows={-Stealth[scale length=2]}, yshift=-2cm] (0,1) -- (1,1);
	\end{tikzpicture}
\end{texlst}
\begin{texlst}
	\begin{tikzpicture}
		\draw [arrows={-Stealth[]}] (0,1) -- (1,1);
		\draw [arrows={-Stealth[scale width=1.5]}, yshift=-1cm] (0,1) -- (1,1);
		\draw [arrows={-Stealth[scale width=2]}, yshift=-2cm] (0,1) -- (1,1);
	\end{tikzpicture}
\end{texlst}

\texinline{arc}选项用来设置箭头中弧形部分:
\begin{texlst}
	\begin{tikzpicture}[ultra thick]
		\draw [arrows={-Hooks[]}] (0,0) -- (1,0);
		\draw [yshift=-1cm, arrows={-Hooks[arc=90]}] (0,0) -- (1,0);
		\draw [yshift=-2cm, arrows={-Hooks[arc=270]}] (0,0) -- (1,0);
	\end{tikzpicture}
\end{texlst}

\texinline{slant}选项用来设置倾斜:
\begin{texlst}
	\begin{tikzpicture}
		\draw [arrows = {->[]}] (0,0) -- (1,0);
		\draw [yshift=-1cm, arrows = {->[slant=0.5]}] (0,0) -- (1,0);
		\draw [yshift=-2cm, arrows = {->[slant=1]}] (0,0) -- (1,0);
	\end{tikzpicture}
\end{texlst}

\texinline{reversed}选项用来反向箭头,并且如果反向两次可以还原:
\begin{texlst}
	\begin{tikzpicture}[ultra thick]
		\draw  [arrows = {-Stealth[reversed]}] (0,0) -- (1,0);
		\draw  [yshift=-1cm, arrows = {-Stealth[reversed, reversed]}] (0,0) -- (1,0);
	\end{tikzpicture}
\end{texlst}

\texinline{harpoon}、\texinline{swap}配合起来实现箭头的左半或右半:
\begin{texlst}
	\begin{tikzpicture}[ultra thick]
		\draw [arrows={-Stealth[harpoon]}] (0,0) -- (1,0);
		\draw [yshift=-1cm, arrows={-Stealth[harpoon, swap]}] (0,0) -- (1,0);
	\end{tikzpicture}
\end{texlst}

\texinline{left}、\texinline{right}是上面两种方法的快捷方式:
\begin{texlst}
	\begin{tikzpicture}
		\draw [arrows={-Stealth[left]}] (0,0) -- (1,0);
		\draw [yshift=-1cm, arrows={-Stealth[right]}] (0,0) -- (1,0);
	\end{tikzpicture}
\end{texlst}

\section{颜色 Color}
同路径一样,箭头也可以设置颜色\texinline{color}及填充\texinline{fill}属性。\texinline{open}属性相当于\texinline{fill=none}。

\section{线条样式}
\subsection{箭头线条帽子 Line Cap}
与线条的类似,但是箭头线条帽子只有两种样式:\forcsvlist{\texinline}{round, butt},\texinline{rect}在这里是不合法的。
\begin{texlst}
	\begin{tikzpicture}[line width=2mm]
		\draw [arrows={-Computer Modern Rightarrow[line cap=butt]}] (0,0) -- (1,0);
		\draw [yshift=-1cm, arrows={-Computer Modern Rightarrow[line cap=round]}] (0,0) -- (1,0);
		\draw [yshift=-2cm, arrows={-Bracket[line cap=butt]}] (0,0) -- (1,0);
		\draw [yshift=-3cm, arrows={-Bracket[line cap=round]}] (0,0) -- (1,0);
	\end{tikzpicture}
\end{texlst}

\subsection{箭头连接 Line Join}
与线条类似,但是箭头连接也只有两种样式:\forcsvlist{\texinline}{round, miter},\texinline{bevel}不合法。
\begin{texlst}
	\begin{tikzpicture}[line width=2mm]
		\draw [arrows={-Computer Modern Rightarrow[line join=miter]}] (0,0) -- (1,0);
		\draw [yshift=-1cm, arrows={-Computer Modern Rightarrow[line join=round]}] (0,0) -- (1,0);
		\draw [yshift=-2cm, arrows={-Bracket[line join=miter]}] (0,0) -- (1,0);
		\draw [yshift=-3cm, arrows={-Bracket[line join=round]}] (0,0) -- (1,0);
	\end{tikzpicture}
\end{texlst}

\subsection{圆角 round}
\texinline{round}属性是\texinline{line cap=round, line join=round}的快捷方式。

\subsection{尖角 sharp}
\texinline{sharp}属性是\texinline{line cap=butt, line join=miter}的快捷方式。

\subsection{线宽 line width}
\texinline{line width}属性用于设置箭头的线宽。

\subsection{扭转伸缩 Beding and Flexing}
前面的全部是连接直线的情况,下面来看看复杂扭曲和伸缩的情况。
\begin{texlst}
	\def\wall{\fill [fill=black!50] (1,-0.5) rectangle (2, 0.5);
		\pattern [pattern=bricks] (1,-0.5) rectangle (2,0.5);
		\draw [line width=1pt] (1cm+0.5pt, -0.5) -- ++(0,1);}
	\begin{tikzpicture}
		\wall
		\draw [red!25, line width=1mm] (-1,0) -- (1,0);
		\draw [red, line width=1mm, -{Stealth[length=1cm, open, blue, quick]}] (-1,-.5) .. controls (0,-.5) and (0,0) .. (1,0);
	\end{tikzpicture}
\end{texlst}
我仔细对比了加\texinline{quick}选项与不加的效果,感觉没有什么区别。
用在一个箭头的情况没有问题,但是在下面的示例中,多个箭头就出问题了。
\begin{texlst}
	\def\wall{\fill [fill=black!50] (1,-0.5) rectangle (2, 0.5);
		\pattern [pattern=bricks] (1,-0.5) rectangle (2,0.5);
		\draw [line width=1pt] (1cm+0.5pt, -0.5) -- ++(0,1);}
	\begin{tikzpicture}
		\wall
		\draw [red!25, line width=1mm] (-1,0) -- (1,0);
		\draw [red, line width=1mm, -{[quick, sep]>>>}] (-1,-.5) .. controls (0,-.5) and (0,0) .. (1,0);
	\end{tikzpicture}
\end{texlst}
这里就开始需要\texinline{bending}库出场了。

\texinline{flex}、\texinline{bend}需要加载\texinline{beding}库。\texinline{flex}以及\texinline{flex'}选项效果一般,我看还不如\texinline{quick
	and dirty},但是\texinline{bend}选项就不一样了,感觉很流畅。
\begin{texlst}
	\usetikzlibrary{arrows.meta, bending}
	\def\wall{\fill [fill=black!50] (1,-0.5) rectangle (2, 0.5);
		\pattern [pattern=bricks] (1,-0.5) rectangle (2,0.5);
		\draw [line width=1pt] (1cm+0.5pt, -0.5) -- ++(0,1);}
	\begin{tikzpicture}
		\wall
		\draw [red!25, line width=1mm] (-1,0) -- (1,0);
		\draw [red, line width=1mm, -{Stealth[length=20pt, bend]}] (-1,-.5) .. controls (0,-.5) and (0,0) .. (1,0);
	\end{tikzpicture}
\end{texlst}

而且在多个箭头的情况下也表现很自然。
\begin{texlst}
	\usetikzlibrary{bending}
	\def\wall{\fill [fill=black!50] (1,-0.5) rectangle (2, 0.5);
		\pattern [pattern=bricks] (1,-0.5) rectangle (2,0.5);
		\draw [line width=1pt] (1cm+0.5pt, -0.5) -- ++(0,1);}
	\begin{tikzpicture}
		\wall
		\draw [red!25, line width=1mm] (-1,0) -- (1,0);
		\draw [red, line width=1mm, -{[bend, sep]>>>}] (-1,-.5) .. controls (0,-.5) and (0,0) .. (1,0);
	\end{tikzpicture}
\end{texlst}

\section{箭头规格}
\subsection{箭头间距 sep}
\begin{texlst}
	\begin{tikzpicture}
		\draw[-{>[sep=1pt]>[sep=2pt]>[sep=5pt]}] (0,1) -- (1,1);
	\end{tikzpicture}
\end{texlst}

\subsection{明确线端}
\begin{texlst}
	\begin{tikzpicture}
		\draw[very thick, <<<->>>] (0,0) -- (2,0);
		\draw[very thick, <.<<->>.>, yshift=-1cm] (0,0) -- (2,0);
		\draw[very thick, <<.<->.>>, yshift=-2cm] (0,0) -- (2,0);
		\draw[very thick, <<.<->.>>, yshift=-3cm] (0,0) to[bend left] (2,0);
	\end{tikzpicture}
\end{texlst}

\subsection{定义箭头快捷方式}
\begin{texlst}
	\begin{tikzpicture}[myarrow/.tip= {Stealth[sep]. >>}]
		\draw[-myarrow] (0,0) -- (2,0);
	\end{tikzpicture}
\end{texlst}

\subsection{>=}
用来设置箭头样式。

\subsection{shorten >=/<=}
设置箭头与目标之间的空距。

\subsection{arrows}
用于设置范围内所有箭头的样式。

\subsection{箭头样式参考}
\begin{arrowexamples}
	\arrowexample Arc Barb[]
	\arrowexample Bar[]
	\arrowexample Bracket[]
	\arrowexample Hooks[]
	\arrowexample Parenthesis[]
	\arrowexample Straight Barb[]
	\arrowexample Tee Barb[]
\end{arrowexamples}

\begin{arrowexamples}
	\arrowexample Classical TikZ Rightarrow[]
	\arrowexample Computer Modern Rightarrow[]
	\arrowexampledouble Implies[]
	\arrowexample To[]
\end{arrowexamples}

\begin{arrowexamples}
	\arrowexample Circle[]
	\arrowexample Diamond[]
	\arrowexample Ellipse[]
	\arrowexample Kite[]
	\arrowexample Latex[]
	\arrowexample Latex[round]
	\arrowexample Rectangle[]
	\arrowexample Square[]
	\arrowexample Stealth[]
	\arrowexample Stealth[round]
	\arrowexample Triangle[]
	\arrowexample Turned Square[]
\end{arrowexamples}

\begin{arrowexamples}
	\arrowexample Circle[open]
	\arrowexample Diamond[open]
	\arrowexample Ellipse[open]
	\arrowexample Kite[open]
	\arrowexample Latex[open]
	\arrowexample Latex[round,open]
	\arrowexample Rectangle[open]
	\arrowexample Square[open]
	\arrowexample Stealth[open]
	\arrowexample Stealth[round,open]
	\arrowexample Triangle[open]
	\arrowexample Turned Square[open]
\end{arrowexamples}

\begin{arrowcapexamples}
	\arrowcapexample Butt Cap[]
	\arrowcapexample Fast Round[]
	\arrowcapexample Fast Triangle[]
	\arrowcapexample Round Cap[]
	\arrowcapexample Triangle Cap[]
\end{arrowcapexamples}

\begin{arrowexamples}
	\arrowexample Rays[]
	\arrowexample Rays[n=8]
\end{arrowexamples}

这一章的内容暂时先到这儿,文档的高级写法真的很迷人。但是现在还是有些看不懂,而且箭头的形状也太多,暂时我只需要了解一些简单的就可以了。

\end{document}
